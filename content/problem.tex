\section{Problembeschreibung}
Der HiveMQ Broker ist, wie in Kapitel \ref{s:hivemq-cluster} beschrieben, clusterfähig und in der Lage zur Laufzeit die grö{\ss}e des Clusters anzupassen.
Um das HiveMQ Cluster für \ac{mqtt} Clients zu einem virtuellen Broker mit einer einzigen \ac{ip} Adresse oder Domain zu abstrahieren, wird ein Load Balancer benötigt. Dieser kann die Last transparent zum Client auf alle Nodes des Clusters verteilen.
In diesem Kapitel werden Anforderung für einen Load Balancer definiert um \ac{mqtt} Traffic klug zu allen Nodes eines HiveMQ Clusters zu verteilen.

\subsection{Cluster Discovery}
Ein HiveMQ Cluster ist, wie in Kapitel \ref{s:hivemq-cluster} beschrieben, in der Lage zur Laufzeit die grö{\ss}e des Clusters anzupassen. Der Load Balancer muss daher ebenfalls in der Lage sein neue HiveMQ Nodes zur Laufzeit zu entdecken und neue Clients mit den neuen Nodes zu verbinden. Falls sich die Anzahl der Nodes verringert, darf der Load Balancer keine neuen Verbindungen mehr mit den alten Nodes aufbauen. Dies würde zu Fehlermeldungen bei den Clients führen.
Eine statische Konfigurierung der Nodes ist somit nicht möglich.

\subsection{Langlebige TCP Verbindungen}
In Kapitel \ref{s:mqtt} wurde erläutert, dass \ac{mqtt} langlebiege \ac{tcp} Verbindungen benutzt um den Datendurchsatz gering zu halten. Dadurch kann der Broker Nachrichten zum Client schicken, ohne das dieser periodisch nach neuen Nachrichten fragen muss.
Dies führt bei einem Update des Load Balancers zu Problemen. Bei \ac{http} kann ein zweiter Load Balancer mit der neuen Version gestartet und als \ac{dns} A Eintrag für die Domain eingetragen werden. Da \ac{http} dem Request / Response Schema folgt sind die Verbindungen kurzlebiger als die von \ac{mqtt}. Folgender Artikel zeigt, wie nach einer Stunde alle \ac{http} Verbindungen zu einem alten Load Balancer abgebaut waren.
%TODO cite this!
Um einen \ac{lb} im Kontext von \ac{mqtt} zu updaten, muss dieser neugestartet werden ohne die aktiven Verbindungen zu terminieren.

\subsection{Persistente Client Session}
Wie in Kapitel \ref{s:persistent-session} beschrieben können Clients eine persistente Session bei einem Broker anfordern. Dabei werden Nachrichten und Session relevante Daten im Arbeitsspeicher gespeichert für den Fall, dass die Verbindung zum Client unterbrochen wird. Sobald der Client seine Verbindung wieder aufbaut, findet ein Client Takeover (\ref{s:client-takeover}) statt.
Dieser Prozess findet bei einem Broker im Arbeitsspeicher statt.
Dort muss zu den existierenden persistierten Daten der dazugehörige \ac{tcp} Socket ausgetauscht werden.
Wenn der Broker jedoch aus einem Cluster besteht, und sich der neue Client auf einem anderen Node als der alte Client verbindet, ist ein Client Takeover komplizierter.
In diesem Szenario muss der Node, auf dem der alte Client verbunden war, alle persistierten Daten an den neuen Node übermitteln.
Wenn der alte Client über einen längeren Zeitraum nicht verbunden war, können sich viele Daten anhäufen.
Persistierte Daten müssen über den Cluster Transport auf den neuen Node transferiert werden. Im Optimallfall wird der neue Client auf dem selben Node verbunden, auf dem er vorher ebenfalls verbunden war, um internen Cluster Traffic zu vermeiden. Durch das verschieben der Daten auf einen anderen Node entstehen zudem Latenzen in der Kommunikation mit dem neuen Client. Die Verbindung kann erst bestätigt werden, wenn alle Daten auf den neuen Node migriert wurden.
\\
Um einen Client immer zum selben Node zu verbinden werden bei anderen Protokollen wie \ac{http} zum Beispiel Cookies verwendet. Die Cookies werden von dem Load Balancer ausgelesen und enthalten Informationen zu welchem Node der Client als letztes verbunden war. Das \ac{mqtt} Protokoll unterstützt keinen Mechanismus, der wie \ac{http} Cookies funktioniert.
Eine weitere Option einen Client immer zum selben Node zu verbinden ist \verb|MAGLEV|. Dabei werden Layer vier Informationen des Clients ausgelesen wie zum Beispiel die \ac{ip} Adresse und gehasht. Anhand des Hashes wird anschlie{\ss}end der Node bestimmt, zu welchem der Client verbunden wird. Solange sich keine der gehashten Informationen ändert, wird der Client immer an den selben Node weitergeleitet. Da \ac{mqtt} auf Layer vier aufsetzt, kann \verb|MAGLEV| verwendet werden.
Je nach Anwendungsfall verbinden sich \ac{mqtt} Clients auch über das Mobilfunknetz. Als Beispiel dienen Lastkraftwagen(LKW), die innerhalb Europas unterwegs sind. Durch den ständigen Standortwechsel wird zwangsweise ebenfalls der Mobilfunkmast gewechselt. Es wird nicht garantiert, dass der LKW immer die selbe \ac{ip} behält. \verb|MAGLEV| würde somit bei einem Verbindungsabbruch den Client an einen neuen Node weiterleiten.
% TODO cite maglev paper
% TODO beispiel für changing ip address in mobile data

\subsection{Ungleiche Lastverteilung}

Es gibt viele verschiedene \ac{mqtt} Clients im Anwedungsfeld \ac{iot}. Manche sind zum Beispiel Temperatursensoren, die alle zehn Minuten einen Integer Wert auf ein Topic veröffentlichen, andere sind Drehzahlsensoren, die alle 500 Millisekunden die aktuelle Drehzahl eines Motors zur Geschwindigkeitsermittlung veröffentlichen.
Die Drehzahlsensoren verursachen durch die erhöhte Frequenz der Nachrichten mehr Traffic und Arbeitslast auf einem Broker.
Bei \textit{Round-robin} und \textit{least connection} Load Balacing Algorithmen führt dieses Verhalten zu einer ungleichen Lastverteilung im Cluster. Dies wird durch folgendes vereinfachtes Beispiel deutlich:
Es gibt neun Temperatursensoren, einen Drehzahlsensor und ein HiveMQ Cluster bestehend aus zwei Nodes. Ein Drehzahlsensor benötigt zehn \ac{cpu} Einheiten und ein Temperatursensor benötigt zwei \ac{cpu} Einheiten. Bei dem Round-robin sowie least connection Load Balancing Algorithmus würden fünf Temperatursensor auf Node1 und vier Temperatursensoren plus ein Drehzahlsensor auf Node2 verbunden sein. Die Gesamtlast auf Node1 beträgt zehn \ac{cpu} Einheiten und auf Node2 achtzehn \ac{cpu} Einheiten. Dieses System hat eine ungleiche Lastverteilung.\\
Dieses Szenario würde bei \ac{http} zu keiner ungleichen Lastverteilung führen. Der Load Balancer kann für jede Nachricht die an das Cluster geschickt wird eine erneute Entscheidung treffen, an welchen Node diese geschickt werden soll. Somit können die Nachrichten des Drehzahlsensors auf beide Nodes aufgeteilt werden und müssen nicht nur von einem Node verarbeitet werden.\\
% TODO vielleicht Grafik?
Bei \ac{mqtt} kann man die Load Balancing Entscheidung nicht auf Nachrichten Basis treffen, wie bei \ac{http}, sondern nur auf Client Ebene. Sobald die \ac{tcp} Verbindung einmal aufgebaut ist, werden alle Nachrichten an den selben Node weitergeleitet.\\
% TODO ich will sagen bei http ist es auf nachrichten ebene
Ein \ac{lb} für \ac{mqtt} muss in der Lage sein, die eingehenden Verbindungen basierend der derzeitigen Arbeitslast der einzelnen Nodes zu gewichten.

% TODO erklären warum least connection nicht ausreicht
% warum least connection bei http geht aber nicht bei mqtt
%https://blog.envoyproxy.io/examining-load-balancing-algorithms-with-envoy-1be643ea121c

\begin{comment}
\ac{mqtt} Clients haben keine Grenze an Topics die sie abonnieren können. Mit einer Wildcard Subscription ist es sogar möglich, alle Topics die auf einem Broker existieren mit einer Subscription zu abonnieren: \verb|#|. Ein Client, der eine solche Subscription hat, verursacht mehr Traffic und Arbeitslast auf einem Broker als ein Client, der alle zehn Minuten einen Integer Wert auf dem Topic \verb|sensors/temperature| veröffentlicht.
Wie in Kapitel \ref{s:domain} erläutert sind im Anwendungsfeld \ac{iot} die beschriebenen Clients keine seltenheit.
% TODO muss ich das begründen?
\end{comment}

\begin{comment}
- Describe insufficiencies your project or thesis aims at. Example: "The automated parts of the claim handling process are susceptible to changes. Each change requires lots of man-ual steps in order to deploy these changes into production."
- Do not take the term "problem" too literal. Sometimes, your project or thesis just aims at im-proving a good status quo or pursues new waysand opportunities that arise due to new technological advances or trends. In this case, describe the status quo and where the op-portunities for improvement are.
- Exemplify things! Describe the problem / status quo by means of a specific scenario with concrete steps, concrete input and output data, etc., supported by expressive figures. Do not be afraid that the reader might think that your solution just works for that particular sce-nario. In general, readers can abstract from concrete details much easier that to envisionconcrete scenarios by interpreting overly generic, vague, meandering text passages. (More-over, writing in vague terms also leaves the impression that the writer either did not under-stand the problem for him- or herself or tries to blow up mundane issues.)
\end{comment}
