\section{Problembeschreibung}
Die clusterfähigkeit des HiveMQ Brokers ermöglicht Millionen \ac{iot} Geräten die Kommunikation untereinander über einen zentralen und ausfallsicheren HiveMQ Cluster.
Um ein HiveMQ Cluster für \ac{mqtt} Clients zu einem virtuellen Broker mit einer einzigen \ac{ip} Adresse oder Domain zu abstrahieren, wird, wie in Kapitel \ref{s:load-balancing} beschrieben, ein \acl{lb} benötigt.
In diesem Kapitel werden Anforderung für einen \acl{lb} definiert, um \ac{mqtt} Traffic klug an alle Nodes eines HiveMQ Clusters transparent zum Client zu verteilen.

\subsection{Cluster Discovery} \label{sp:cluster-discovery}
Ein HiveMQ Cluster ist, wie in Kapitel \ref{s:hivemq-cluster} beschrieben, in der Lage, zur Laufzeit die grö{\ss}e des Clusters anzupassen. Der Load Balancer muss daher ebenfalls in der Lage sein, neue HiveMQ Nodes zur Laufzeit zu entdecken und neue Clients mit den neuen Nodes zu verbinden. Falls sich die Anzahl der Nodes verringert, darf der Load Balancer keine neuen Verbindungen mehr mit den alten Nodes aufbauen. Dies würde zu Fehlermeldungen bei den Clients führen.
Eine statische Konfiguration der HiveMQ Nodes ist somit nicht möglich.

\subsection{Langlebige TCP Verbindungen}
TODO Abschnitt vielleicht streichen
\\
In Kapitel \ref{s:mqtt} wurde erläutert, dass \ac{mqtt} langlebige \ac{tcp} Verbindungen benutzt um den Datendurchsatz gering zu halten. Dadurch kann der Broker Nachrichten zum Client schicken, ohne das dieser periodisch nach neuen Nachrichten fragen muss.
Der Load Balancer muss aktive Client Verbindungen solange aufrecht halten, bis Clients diese terminieren.
Aktualisierungen des Load Balancers oder dessen Konfiguration dürfen nicht zu einer Unterbrechung der aktiven Verbindungen führen.

\subsection{Persistente Client Sessions}
Wie in Kapitel \ref{s:persistent-session} beschrieben können Clients eine persistente Session bei einem Broker anfordern. Dabei werden Nachrichten und Session relevante Daten gespeichert für den Fall, dass die Verbindung zum Client unterbrochen wird. Sobald der Client seine Verbindung wieder aufbaut, findet ein Client Takeover (\ref{s:client-takeover}) statt.
Dieser Prozess findet bei einem Broker im Arbeitsspeicher statt.
Dort muss zu den existierenden persistierten Daten der dazugehörige \ac{tcp} Socket ausgetauscht werden.
Wenn der Broker jedoch aus einem Cluster besteht, und sich der neue Client auf einem anderen Node als der alte Client verbindet, ist ein Client Takeover komplexer.
In diesem Szenario muss der Node, auf dem der alte Client verbunden war, alle persistierten Daten an den neuen Node übermitteln.
Wenn der alte Client über einen längeren Zeitraum nicht verbunden war, können sich viele Daten anhäufen.
Im Optimallfall wird der neue Client mit dem selben Node verbunden, mit dem er zuvor verbunden war, um internen Cluster Traffic zu vermeiden. Durch das Verschieben der Daten auf einen anderen Node entstehen zudem Latenzen in der Kommunikation mit dem neuen Client. Die Verbindung kann erst bestätigt werden, wenn alle Daten auf den neuen Node migriert wurden.
\begin{comment}
Um einen Client immer zum selben Node zu verbinden werden bei anderen Protokollen wie \ac{http} zum Beispiel Cookies verwendet. Die Cookies werden von dem Load Balancer ausgelesen und enthalten Informationen zu welchem Node der Client als letztes verbunden war. Das \ac{mqtt} Protokoll unterstützt keinen Mechanismus, der wie \ac{http} Cookies funktioniert.
Eine weitere Option einen Client immer zum selben Node zu verbinden ist consistent hashing, wie in Kapitel \ref{sb:lb-algo} beschrieben. Dabei werden Layer vier Informationen des Clients ausgelesen, wie zum Beispiel die \ac{ip} Adresse, und gehasht. Anhand des Hashes wird anschlie{\ss}end der Node bestimmt, zu welchem der Client verbunden wird. Solange sich keine der gehashten Informationen ändert, wird der Client immer an den selben Node weitergeleitet.
Je nach Anwendungsfall verbinden sich \ac{mqtt} Clients auch über das Mobilfunknetz. Durch den ständigen Standortwechsel wird zwangsweise ebenfalls der Mobilfunkmast gewechselt. Es wird nicht garantiert, dass der LKW immer die selbe \ac{ip} behält. \verb|MAGLEV| würde somit bei einem Verbindungsabbruch den Client an einen neuen Node weiterleiten.
% TODO cite maglev paper
% TODO beispiel für changing ip address in mobile data
\end{comment}

\subsection{Circuit Breaking}
In Kapitel \ref{sb:overload-protection} wurde die Overload Protection eines HiveMQ Brokers erläutert. Dies ermöglicht dem Broker die Verbindung individueller Clients, die viel Arbeitslast auf dem Broker erzeugen, zu unterbrechen. Falls der \ac{mqtt} Client sich \ac{mqtt} Konform verhält, versucht dieser die Verbindung automatisch wiederherzustellen.
In diesem Fall darf der Load Balancer den Client nicht wieder mit dem selben Node verbinden, da dieser Node mit dem Client erneut überlastet sein wird. Es muss ein Node mit einer geringeren Auslastung für diesen Client gefunden werden.
\\
Das Overload Protection Level gibt an, ob sich der HiveMQ Node in einer Überlast befindet. Der Load Balancer sollte den Node im Fall einer Überlast unterstützen indem er keine neuen Clients mehr mit diesem Node verbindet.

\subsection{Ungleiche Lastverteilung} \label{sp:load}
Es gibt viele verschiedene \ac{mqtt} Clients im Anwedungsfeld \ac{iot}. Manche sind zum Beispiel Temperatursensoren, die alle zehn Minuten einen Integer Wert auf ein Topic veröffentlichen, andere sind Drehzahlsensoren, die alle 500 Millisekunden die aktuelle Drehzahl eines Motors zur Geschwindigkeitsermittlung veröffentlichen.
Die Drehzahlsensoren verursachen durch die erhöhte Frequenz der Nachrichten mehr Traffic und Arbeitslast auf einem Broker als die Temperatursensoren.
\\
Da \ac{mqtt} ein Verbindungsorientiertes Protokoll ist, kann die load balancing Entscheidung nicht auf Paket- sondern nur auf Verbindungsebene getroffen werden. Sobald eine Client Verbindung aufgebaut ist, werden alle Nachrichten des Clients immer an den selben Node weitergeleitet.
Aus diesem Grund kann bei einem \textit{round-robin} oder \textit{least connection} load balancing Algorithmus das zuvor beschriebene Verhalten zu einer ungleichen Lastverteilung im Cluster führen.
\\
Ein \ac{lb} für \ac{mqtt} muss in der Lage sein, die eingehenden Verbindungen basierend der derzeitigen Arbeitslast der einzelnen Nodes zu gewichten.

\begin{comment}
\ac{mqtt} Clients haben keine Grenze an Topics die sie abonnieren können. Mit einer Wildcard Subscription ist es sogar möglich, alle Topics die auf einem Broker existieren mit einer Subscription zu abonnieren: \verb|#|. Ein Client, der eine solche Subscription hat, verursacht mehr Traffic und Arbeitslast auf einem Broker als ein Client, der alle zehn Minuten einen Integer Wert auf dem Topic \verb|sensors/temperature| veröffentlicht.
Wie in Kapitel \ref{s:domain} erläutert sind im Anwendungsfeld \ac{iot} die beschriebenen Clients keine seltenheit.
% TODO muss ich das begründen?
\end{comment}

\newpage

\begin{comment}
- Describe insufficiencies your project or thesis aims at. Example: "The automated parts of the claim handling process are susceptible to changes. Each change requires lots of man-ual steps in order to deploy these changes into production."
- Do not take the term "problem" too literal. Sometimes, your project or thesis just aims at im-proving a good status quo or pursues new waysand opportunities that arise due to new technological advances or trends. In this case, describe the status quo and where the op-portunities for improvement are.
- Exemplify things! Describe the problem / status quo by means of a specific scenario with concrete steps, concrete input and output data, etc., supported by expressive figures. Do not be afraid that the reader might think that your solution just works for that particular sce-nario. In general, readers can abstract from concrete details much easier that to envisionconcrete scenarios by interpreting overly generic, vague, meandering text passages. (More-over, writing in vague terms also leaves the impression that the writer either did not under-stand the problem for him- or herself or tries to blow up mundane issues.)
\end{comment}
