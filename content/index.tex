Guidelines:
Seiten: 50
Präsentation: 30 min / Disskusion: 15 min
sonarqube als code scan

\section{Einleitung}

\subsection{Motivation}
\subsection{Unternehmen}
\subsection{Zielsetzung}
vielleicht im kapitel motivation?
\subsection{Aufbau der Arbeit}

- Outline your work such that readers who have not read subsequent chapters get an idea of what
  - your "problem domain" is (e.g. the software quality department of some company or the overarching research project you are working in)
  - the situation is (e.g. software tests at some company are done in complete manual fash-ion)
  - the complication / problem is (e.g. hotfixes have to be deployed totally untested in pro-duction)
  - what your approach is (e.g. establishing a concept and tool evaluation for continuously improving code coverage of automated tests)
  - what beyond the scope of your thesis is (e.g. also implementing a CICD pipeline)
  - what the actual core results are (e.g. how many tools have been evaluated)

From that it must clear, what your thesis'contribution actually is. Please delineate your contribution from every concept, method, tool, framework or whatever implementation your thesis took for granted and is built upon.

This chapter deliberately anticipates contents of the latter chapters but please refrain from spe-cific technical termsof your application domain or solution here. Trade technical accuracy for com-mon comprehensibility here. Write this chapter as a "management summary" and do not spend more than two to three pages here.

Also explain structure of your thesis, i.e., summarize the contents of each chapter in one sentence or paragraph and describe the overarching storyline, i.e., how chapters build upon each other.

\section{Anwendungsfeld Internet of Things}

ein grobes beispiel einet mqtt broker -> client anwendungs beschreiben
ein / mehrere broker (mehrere millionen clients zb MAN, darf ich das erzählen ?)
sehr viele iot clients
big downstream clients -> backends
industrial bereich

- Describe the (business) domain of your work. Ask yourself: What does the common reader need to know about the domain in order to understand the results of your work? For example, if you work contributes to the claim handling process of some insurance company, describe the common claim handling process.
- Please refrain from meandering explanations about details that have no relevance for later chapters (just in order to increase your page count).
- The header "Domain / Anwendungsfeld" is supposed to be replaced or extended with some more specific header like: "Domain 'Claim handling'"

\section{Technische Grundlagen} - Technologische Grundlagen - Grundlagen /

\subsection{MQTT}
\acf{mqtt} wurde ursprünglich von Doktor Andy Stanford-Clark und Arlen Nipper im Jahr 1999 entworfen um Gas- und Ölpiplines zu überwachen. Diese lagen oftmals an entlegenen Orten, wie zum Beispiel auf Übersee, und konnten nicht mit Radiowellen oder einem Kabel zum Festland erreicht werden. Zu dieser Zeit war die einzige Option um Sensordaten auf einen Server zu übertragen eine auf Datendurchsatz abgerechnete Satellitenkommunikation. Bei mehreren tausend Sensoren wurde somit ein Protokoll benötigt, das die Daten zuverlässig mit minimaler Bandbreite an die Server auf dem Festland übermitteln kann.
\ac{mqtt} wurde im Jahr 2013 von der \ac{oasis} als Open Source standatisiert und wird heutzutage von vielen gro{\ss}en \ac{iot} Platformen unterstützt.\cite{WhatMQTTDefinition}\\
Es gibt derzeit zwei Versionen der MQTT Spezifikation: \verb|3.1.1| und \verb|5|. Alle Referenzen, falls nicht explizit gekennzeichnet, beziehen sich auf die aktuelle Version \verb|5| des Protokolls.\\
\ac{mqtt} ist ein Layer 7 \textit{Publish and Subscribe} Protkoll, das auf \acs{tcp} / \acs{ip} aufsetzt. Anders als das Request / Response Paradigma bei \acs{http} ist \ac{mqtt} Event gesteuert und erlaubt Nachrichten direkt an einen bestimmten Client zu schicken. Somit muss nicht periodisch nach neuen Nachrichten gefragt werden. Zusammen mit einem fixen Paket-Header von nur zwei Byte sorgen diese Eigeschaften für einen minimalen Datendurchsatz.\cite{WhatMQTTDefinition}\\

\subsubsection{Publish and Subscribe}
\ac{mqtt} nutzt das \textit{Publish and Subscribe} Kommunikationsschema, das eine Struktur bietet um Nachrichten zwischen Herausgeber und Abonnent auszutauschen. In diesem Schema gibt es zwei unterschiedliche Systeme:
\begin{itemize}
    \item Clients
    \item Broker
\end{itemize}
Clients sind alle Teilnehmer dieses Systems, die Nachricht empfangen oder veröffentlichen wollen. Diese setzten auf einen Broker als Mittelsmann um die Nachrichten erfolgreich zu vermitteln.\cite{teamGettingStartedMQTT} Jede Nachricht wird durch den Herausgeber in bestimmte Klassen kategorisiert. Der Herausgeber wei{\ss}t nicht ob, oder wer, an dieser Nachricht interessiert ist und schickt die klassifizierte Nachricht an den Broker. Clients, die an bestimmten Nachrichten interessiert sind, müssen bei dem Broker ein Abonnement erstellen und dabei die gewünschte Kategorie angeben. Somit kann der Broker eingehende kategorisierte Nachrichten an die entsprechenden Clients weiterleiten. Der Broker kann Nachrichten zudem auch aufbewahren, sodass Clients, die zu einem späteren Zeitpunkt auf ein Thema abonnieren, ebenfalls die vergangenen Nachrichten erhalten. Durch dieses System entsteht eine Entkopplung der einzelnen Clients auf mehreren Ebenen.\cite{EverythingYouNeed}
\begin{itemize}
    \item \textbf{Räumliche Entkopplung:} Herausgeber und Abonnent der Nachricht müssen sich nicht kennen.
    \item \textbf{Zeitliche Entkopplung:} Herausgeber und Abonnent müssen nicht zur selben Zeit aktiv sein.
    \item \textbf{Synchronisierungs Entkopplung:} Herausgeber und Abonnent müssen ihre Operationen beim publizieren und konsumieren nicht unterbrechen. TODO warum?:
\end{itemize}
\cite{teamPublishSubscribeMQTT}
In verteilten System, wo einzelne Komponenten oftmals sehr unterschiedlich sind, spielt ein solches Konzept eine gro{\ss}e Rolle, da es eine einheitliche Abstraktion der Kommunikationsebene bietet.\cite{domingusDistributedSystemsIntroduction2020}\\
% TODO mehr zu distributed systems
% TODO bild publisher - subscriber topic
% TODO bild von sensoren / broker / downstream clients
\begin{figure}
    \centering
    %\includegraphics[scale=0.5]{images/integration-app.png}
    \caption{Publish and Subscribe Architektur basirend auf Themen}
    \label{fig:publish-subscribe}
\end{figure}

Im Kontext \ac{mqtt} werden Nachrichten in eine hieraschich aufgebaute Themenstruktur klassifiziert. Themen werden mit einem Schrägstrich (\verb|/|) getrennt und sehen zum Beispiel wie folgt aus: \verb|sensors/temperature/celcius|. Abbildung \ref{fig:publish-subscribe} zeigt einen Client, der auf das Thema \verb|bla| Nachrichten veröffentlicht. Diese wird an zwei weitere Clients durch den Broker weitergeleitet, da diese das Thema \verb|bla| abonniert haben. Themen müssen auf einem Broker nicht explizit angelegt werden. Sobald ein Client auf einem Thema publiziert oder es abonniert, wird dieses automatisch angelegt.\cite{WhatMQTTDefinition}\\
Beim abonnieren eines Themas kann entweder ein spezifisches Thema oder eine Kombination aus Thema und Wildcard verwendet werden. Bei einer Wildcard werden zwischen den folgenden zwei unterschieden:\cite{mqtt5Specification}
\begin{itemize}
    \item Multi-level '\verb|#|': Schlie{\ss}t das Vorgänger- und alle nachfolgenden Themen mit ein.
    \item Single-level '\verb|+|': Schlie{\ss}t alle Themen auf einer einzigen Ebene mit ein.
\end{itemize}
Bei der folgenden Themenstruktur
\begin{itemize}
    \item \verb|sensors|
    \item \verb|sensors/temperature|
    \item \verb|sensors/temperature/celcius|
    \item \verb|sensors/temperature/kelvin|
    \item \verb|sensors/fuel|
    \item \verb|sensors/fuel/tank1|
    \item \verb|sensors/fuel/tank2|
\end{itemize}
schlie{\ss}t ein Abonnement auf \verb|sensors/#| alle Themen mit ein. Bei einem Abonnement auf \verb|sensors/+| sind hingegen nur diese Themen mit eingeschlossen:
\begin{itemize}
    \item \verb|sensors/temperature|
    \item \verb|sensors/fuel|
\end{itemize}

\subsubsection{Quality of Service}
Je zuverlässiger eine Nachricht übermittelt werden soll, desto mehr Datendurchsatz verursacht diese Nachricht im gesamten System. Wenn ein Client wissen will, ob seine Nachricht im Broker eingetroffen ist, muss der Broker dem Client eine entsprechende Rückmeldung geben. Andernfalls kann der Client seine Nachricht an den Broker schicken ohne eine Rückmeldung zu erwarten. Dies ist vergleichsweise im \ac{http} nicht möglicht. Dort muss auf jede Nachricht geantwortet werden.\\
Bei \ac{mqtt} wird die Zuverlässigkeit der Übermittlung einer Nachricht mit einem \textit{\acf{qos}} Level festgelegt. Eine Nachricht publizierte Nachricht muss einen der drei \ac{qos} Level haben:
\begin{itemize}
    \item \ac{qos} 1: Maximal eine Zustellung der Nachricht.
    \item \ac{qos} 2: Mindestens eine Zustellung der Nachricht.
    \item \ac{qos} 3: Genau eine Zustellung der Nachricht.
\end{itemize}
Bei \ac{qos} Level 2 und 3 werden Handshakes zur Verifizierung der Paketübermittelung eingesetzt.\cite{mqtt5Specification}

\subsubsection{Paket Struktur}
\ac{mqtt} hat die Absicht ein leichtgewichtiges Protokoll zu sein. Tabelle \ref{table:mqtt-packet-structure} zeigt den Aufbau eines jeden \ac{mqtt} Paketes. Die fixe Kopfzeile ist zwei Byte gro{\ss} und muss in jedem Paket vorhanden sein. Basierend auf der Art des Paketes, das in der fixen Kopfzeile angegeben wird, sind zusätzlich eine variable Kopfzeile und weitere Daten möglich.\cite{mqtt5Specification}
\begin{table}[h!]
\centering
\renewcommand{\arraystretch}{1.5}
\begin{tabular}{|c|}
    \hline
    Fixe Kopfzeile, muss in jedem \ac{mqtt} Paket vorhanden sein \\
    \hline
    Variable Kopfzeile, optional \\
    \hline
    Daten des Pakets, optional \\
    \hline
\end{tabular}
\caption{Struktur eines \ac{mqtt} Paketes}
\label{table:mqtt-packet-structure}
\end{table}
Tabelle \ref{table:fixed-header} zeigt den detaillierten Aufbau der fixen Kopfzeile. Im ersten Byte werden Bit sieben bis vier für die spezifische Art des Paketes verwendet. Tabelle \ref{table:mqtt-packet-types} enhält eine Liste mit allen möglichen Pakettypen und deren Wert. Byte zwei gibt die restliche Paketlänge encodiert als \textit{Variable Byte Integer} an. Ein \textit{Variable Byte Integer} ist ... . Somit ist eine maximale Paketgrö{\ss}e von xyz möglich.\cite{mqtt5Specification}
% TODO was ist ein variable byte integer
\begin{table}[h!]
\centering
\renewcommand{\arraystretch}{1.5}
\begin{tabularx}{\textwidth}{|c| *{8}{Y|}}
    \hline
    Bit & 7 & 6 & 5 & 4 & 3 & 2 & 1 & 0 \\
    \hline
    \hline
    Byte 1 & \multicolumn{4}{c|}{\ac{mqtt} Paketart} & \multicolumn{4}{c|}{Paketart spezifische Kennzeichnung} \\
    \hline
    Byte 2 & \multicolumn{8}{c|}{Restliche Paketlänge} \\
    \hline
\end{tabularx}
\caption{Aufbau der fixen \ac{mqtt} Kopfzeile}
\label{table:fixed-header}
\end{table}

\begin{table}[h!]
\centering
\renewcommand{\arraystretch}{1.5}
\begin{tabular}{|c|c|}
    \hline
    \textbf{Name} & \textbf{Wert} \\
    \hline
    \hline
    Reserved & 0 \\
    \hline
    CONNECT & 1 \\
    \hline
    CONNACK & 2 \\
    \hline
    PUBLISH & 3 \\
    \hline
    PUBACK & 4 \\
    \hline
    PUBREC & 5 \\
    \hline
    PUBREL & 6 \\
    \hline
    PUBCOMP & 7 \\
    \hline
    SUBSCRIBE & 8 \\
    \hline
    SUBACK & 9 \\
    \hline
    UNSUBSCRIBE & 10 \\
    \hline
    UNSUBACK & 11 \\
    \hline
    PINGREQ & 12 \\
    \hline
    PINGRESP & 13 \\
    \hline
    DISCONNECT & 14 \\
    \hline
    AUTH & 15 \\
    \hline
\end{tabular}
\caption{Verfügbare \ac{mqtt} Paketarten und deren Wert}
\label{table:mqtt-packet-types}
\end{table}

\subsubsection{MQTT CONNECT}

\subsubsection{Shared Subscriptions}


\newpage

\subsection{HiveMQ Broker}
\subsubsection{HiveMQ Cluster}
\subsection{Load Balancing}
was für load balancer gibt es ?
\subsection{Envoy}
vielleicht hier noch kein envoy ? ist dies grundlage oder teil der arbeit das wir uns für envoy entscheiden ?

If your work makes use of existing non-standard systems, tools, frameworks, librariesetc. describe these as well. However, just give a broad overview and delve just into those details that are crucial for the understanding of the following chapters. For standard sys-tems, tools, frameworks, and libraries please refer to its authoritative documentation. Youare not supposed to give an thorough introduction into, e.g., Java or ReactJS.

\section{Problembeschreibung}

\subsection{Cluster Discovery}
discovery that works with hivemq cluster discovery (dns)
es kommen immer wieder neue nodes hinzu / weg zb bei rolling updates
load balancer muss diese nodes dynamisch hinzufüger oder entfernen
hivemq ist in der lage dynamische cluster joins zu machen also muss der lb auch in der lage dazu sein

\subsection{Langlebige TCP Verbindungen}
ein LB darf die tcp connections nicht terminieren bei einem update oder config änderung
mqtt setzt auf langlebiege tcp verbindungen! nicht wie bei http! bei http -> drain
-> ein drain bei mqtt kann tage dauern

load balancer with data plane api -> good docs
\subsection{Persistent Client Session}
mqtt clients haben viel session auf dem broker
bei einem reconnect wird ein client takeover ausgeführt -> teuer
clients möglichst immer zum selben broker routen
problem: iot clients haben oft sich ändernde ip adressen zb mobile geräte wie autos
\subsubsection{Client Takeover}
sehr teuer! die persistent client informationen müssen umgezogen werden auf neuen broker

\subsection{Ungleiche Lastverteilung}
mqtt clients sind unterschiedlich teuer (nicht wie bei http)
beispiel: wildcard subscription

\subsubsection{Downstream Clients}
\subsubsection{Lightweight Clients}
\subsubsection{Wildcard Subscriptions}

- Describe insufficiencies your project or thesis aims at. Example: "The automated parts of the claim handling process are susceptible to changes. Each change requires lots of man-ual steps in order to deploy these changes into production."
- Do not take the term "problem" too literal. Sometimes, your project or thesis just aims at im-proving a good status quo or pursues new waysand opportunities that arise due to new technological advances or trends. In this case, describe the status quo and where the op-portunities for improvement are.
- Exemplify things! Describe the problem / status quo by means of a specific scenario with concrete steps, concrete input and output data, etc., supported by expressive figures. Do not be afraid that the reader might think that your solution just works for that particular sce-nario. In general, readers can abstract from concrete details much easier that to envisionconcrete scenarios by interpreting overly generic, vague, meandering text passages. (More-over, writing in vague terms also leaves the impression that the writer either did not under-stand the problem for him- or herself or tries to blow up mundane issues.)

\section{Vobereiten und verwandte Arbeiten}
vielleicht brauche ich diese sektion gar nicht ?

\subsection{Load Balancer}
was für load balancer gibt es? was haben diese für eigenschaften?

\subsection{Shared Subscriptions}
diese lösen das problem der teuren clients ABER clients müssen dann möglichst optimal auf die broker verteilt werden
-> load balancer

iot mqtt threat model ? maybe show this ?

- If your work is based on preliminary work, then outline this preliminary work: "The de-ployment into production is itself semi-automatic. In a continuous integration pipeline..."
- Do some research forrelated work, e.g., commercial products, research prototypes or con-ceptual research that have the same or similar objectives like your work. Compare and de-lineate your work with/from the related work.
- Keep an eye on the proper citation of the works you are describing
- Conclude this chapter with some insufficiencies or shortcomings of the preliminary or related work. This should motivate the necessity of your work.

\section{Lösungskonzept}
nur das konzept WIE ich das lösen will zb. man kann wie clientid parsen mit einem WASM modul
in der nächsten section zeige ich dann den eigentlichen code welcher dies tut

\subsection{Envoy}
\subsubsection{Data-Plane API}
\subsubsection{Hot Restart}

\subsection{DNS Service Discovery}

\subsection{Weighted Round Robin}
\subsubsection{Overload Protection}
\subsubsection{Client Credits}
\subsubsection{Global Tasks}

\subsection{Sticky Session with Client ID}
\subsubsection{MQTT CONNECT}
\subsubsection{Envoy WASM Network Filter}

- This is supposed to be the core of your thesis or project. Describe your work from a con-ceptual viewpoint.
- Example: In case you have developed some prototypical tool in your bachelor thesis, demonstrate how it is employed in its business context. More concrete example: Assume that your contribution is a Maven-Build-Plugin that further automates the deployment of changes to the claim handling process into production. In this case show how the plugin is integrated in the overall (continuous) integration and deployment process, which human ac-tors are involved, which external systems and so on. Elaborate on subtle edge cases you had to deal with, e.g., possible outages of external systems.
- Usedi  agramswhere appropriate. Standard notations are better than informal box-and-line-diagrams. Typical standard notations for a solution concept are
  - Business Process Modelling Notation (BPMN) or UML activity diagrams, that depict a workflow in which your tool is used
  - UML component diagrams, where your tool is represented by just a single component (without its ingredients) together with connected external systems

\section{Architektur und Implementation}

\subsection{Envoy}
\subsubsection{Golang Data-Plane}
\subsubsection{DNS Service Discovery}
\subsubsection{HiveMQ Metriken}
\subsubsection{Weighted Round Robin}
\subsubsection{ClientID Network Filter}

\subsection{Deployment}
\subsubsection{HiveMQ}
\subsubsection{Envoy}

- Describe the realization of your concepts, in case you have actually developed some-thing.
- Elaborate on the software architecture of your tool, in case you have developed one. Use nested UML packages, components, and interfaces in a component diagram.
- If applicable, show the deploymentof your tool in a production environment. Use UML's deployment diagram notation.
- It must be clear from the architecture, what your thesis contributes and what it takes for granted like existing systems, code bases, libraries and frameworks. For example, you can decorate the components in a UML component diagram that you have implemented and those that you just used.
- Do not delve into ordinary details by, e.g., intensively describing a "p lain old Java-object (POJO)" in all its dreary getter-setter-details. Instead pick some interesting details and de-scribe them, e.g.,
  - if   you made extensive use of a certain design pattern, describe a single concrete appli-cation of it using, e.g., UML class diagrams or
  - if your work involves a complicated conversation pattern or protocol, explained it using a UML sequence diagram, state chart, activity diagram and the like or
  - if you have developed a central and canny algorithm, you may even show its implemen-tation in, e.g., Java code.
- Show how the result of your work actually looks like. In case of a tool, provide some screenshots together with some explanatory text.
- Describe the quantity of your work, e.g., in terms of lines of code or classes etc. Please just count your own hand-crafted code but not previously existing, imported or generated code.
- Describe the quality of your work, e.g., if you have developed a large web application, run some performance tests, depicts results and draw conclusions from them.

\section{Erprobung}

\subsection{Lastverteilung}

If possible, do an (external) evaluation of your work. If you have developed some kind of tool, let us-ers test it, gather their feedback and describe that here. (Negative feedback will not contribute to a downgrading).

\section{Zeitplanung und Arbeitspakete}

- In the initiation phase, we agree on work packages. Give a (tabular overview) of which work packages have been finalized
  - to what degree
  - and in which time frame
- Please described the unforeseen difficulties that resulted in unfinished or abandoned work packages


\section{Ausblick}

- Again, summarize your work. This time you can assume that reader have read the rest of the document, i.e., you are free to use even domain-specific terms.
- In case of a master thesis in "Technische Informatik", you will have to provide an addi-tional "technical report" in English of 4-8 pages (cf. examination regulation document, §28, 1d). It is okay for me if you use this technical report as the summary here.
- Usually, during your project or thesis new and extended questions arise that are not dealt with in your project or thesis due effort reasons. Please delineate these in the outlook.

BEISPIELE:

Figure \ref{fig:mender-integration} shows all microservices and their network connections.
\begin{figure}
    \centering
    \includegraphics[scale=0.5]{images/integration-app.png}
    \caption{Mender Integration Server Architecture}
    \label{fig:mender-integration}
\end{figure}
Minio is a third-party object storage. It can either be used to serve uploaded content on its own or to proxy requests to Amazon S3 compatible cloud providers. All other services are mender application logic web services.
\newpage

\begin{figure}
    \import{gen/}{example}
    \caption{Example Code Listing}
    \label{code:example-label}
\end{figure}

Listing \ref{code:example-label} is a very good YAML file.
