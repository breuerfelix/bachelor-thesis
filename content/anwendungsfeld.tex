\section{Anwendungsfeld Internet of Things} \label{s:domain}
Das Internet hat die Welt der Computer und der Kommunikation revolutioniert. Es ist ein Medium für eine geographisch unabhängige Kollaboration und Interaktion zwischen Mensch und Computer.
\cite{BriefHistoryInternet}
\\
Das Internet der Dinge (engl. \textit{\ac{iot}}) nutzt dieses Medium um nicht nur Mensch und Computer zu vernetzen, sondern auch Objekte. Es ist eine Grundlage um Sensoren, Aktoren, Smartphones, Maschinen, Lampen, Autos oder andere elektronischen Geräte mit dem Internet zu verbinden.
\cite{morganSimpleExplanationInternet}
Dadurch wird eine Kommunikation zwischen Mensch zu Objekt oder Objekt zu Objekt ermöglicht.
\cite{uckelmannArchitectingInternetThings2011}
Solche Objekte werden auch \textit{Smart Devices} genannt.
\textit{Gartner} \cite{hungGartnerInsightsHow} und \textit{statista} \cite{GlobalIoTNonIoT} erwarten bis 2025 rund 30 - 75 Millionen Geräte die mit dem Internet verbunden sind.
\\
Durch diese Vernetzung alltäglicher Objekte wird unsere Umgebung klüger und reaktionsfähig. Als Beispiel können Kaffeemaschinen automatisch Kaffee nachbestellen oder Maschinen einen Techniker anfordern sobald ein Defekt vorliegt.
\cite{rangerWhatIoTEverything}
\\
Die Idee verschiedene Objekte mit dem Internet zu verbinden, wurde bereits um 1980 und 1990 behandelt, jedoch waren zu dieser Zeit die Mikrocontroller und Mikrochips noch zu gro{\ss} und zu langsam um effektiv eingesetzt werden zu können.
Mit der stetigen Optimierung der Chips werden immer mehr Objekte mit diesen ausgestattet und sind in der Lage mit ihrer Umwelt zu kommunizieren.
\cite{rangerWhatIoTEverything}
\\
Die stetige Zunahme von Smart Devices stellt jedoch auch aktuelle Kommunikationsprotokolle und Infrastrukturen vor neue Herausforderungen.
Diese sind hauptsächlich für statische Geräte mit einer stetigen Strom- und Internetanbindung entwickelt worden. Millionen Geräte, die unteranderem auch durch ein instabiles mobiles Netzwerk mit dem Internet verbunden sind, erfordern neue Protokolle und Technologien um sich optimal in die bestehende Infrastruktur zu integrieren.
\cite{uckelmannArchitectingInternetThings2011}
\newpage

\begin{comment}
ein grobes beispiel einet mqtt broker -> client anwendungs beschreiben
ein / mehrere broker (mehrere millionen clients zb MAN, darf ich das erzählen ?)
sehr viele iot clients
big downstream clients -> backends
industrial bereich

- Describe the (business) domain of your work. Ask yourself: What does the common reader need to know about the domain in order to understand the results of your work? For example, if you work contributes to the claim handling process of some insurance company, describe the common claim handling process.
- Please refrain from meandering explanations about details that have no relevance for later chapters (just in order to increase your page count).
- The header "Domain / Anwendungsfeld" is supposed to be replaced or extended with some more specific header like: "Domain 'Claim handling'"
\end{comment}
