\section{Anwendungsfeld Internet of Things} \label{s:domain}
Das Internet hat die Welt der Computer und der Kommunikation revolutioniert. Es ist ein Medium für eine geografisch unabhängige Kollaboration und Interaktion zwischen Mensch und Computer.
\cite{BriefHistoryInternet}
\\
Das Internet der Dinge nutzt dieses Medium, um nicht nur Mensch und Computer zu vernetzen, sondern auch Objekte.
Es bildet eine Grundlage, um Sensoren, Aktoren, Smartphones, Maschinen, Lampen, Autos oder andere elektronischen Geräte mit dem Internet zu verbinden \cite{kranenburgInternetThingsCritique2008}.
Dadurch wird eine Kommunikation zwischen Mensch und Objekt oder Objekt und Objekt ermöglicht \cite[S.~2]{uckelmannArchitectingInternetThings2011}.
Solche Objekte werden auch \textit{Smart-Devices} genannt.
Gartner \cite{hungGartnerInsightsHow} und statista \cite{GlobalIoTNonIoT} erwarten bis 2025 rund 30 - 75 Milliarden Geräte, die mit dem Internet verbunden sind.
\\
Durch die Vernetzung alltäglicher Objekte wird unsere Umgebung klüger und reaktiv.
Die Idee, verschiedene Objekte mit dem Internet zu verbinden, wurde bereits um 1980 und 1990 behandelt, jedoch waren zu dieser Zeit die Mikrocontroller und Mikrochips zu gro{\ss} und zu langsam, um effektiv eingesetzt werden zu können.
Mit der stetigen Optimierung der Chips können immer mehr Objekte mit solchen ausgestattet und somit in die Lage versetzt werden, mit ihrer Umwelt zu kommunizieren.
\cite{rangerWhatIoTEverything}
\\
Die stetige Zunahme von Smart-Devices stellt jedoch aktuelle Kommunikationsprotokolle und Infrastrukturen vor neue Herausforderungen.
Diese wurden ursprünglich für statische Geräte mit einer stetigen Strom- und Internetanbindung entwickelt. Millionen Geräte, die unter anderem auch durch ein instabiles mobiles Netzwerk mit dem Internet verbunden sind, erfordern neue Protokolle und Technologien, die ihnen eine optimale Integration in die bestehende Infrastruktur ermöglichen.
\cite[S.~7f]{uckelmannArchitectingInternetThings2011}
\newpage
