\section{Anwendungsfeld Internet of Things} \label{s:domain}
Das Internet hat die Welt der Computer und der Kommunikation revolutioniert. Es ist ein Medium für eine geografisch unabhängige Kollaboration und Interaktion zwischen Mensch und Computer.
\cite{BriefHistoryInternet}
\\
Das Internet der Dinge nutzt dieses Medium, um nicht nur Mensch und Computer zu vernetzen, sondern auch Objekte.
Es bildet eine Grundlage um Sensoren, Aktoren, Smartphones, Maschinen, Lampen, Autos oder andere elektronischen Geräte mit dem Internet zu verbinden \cite{morganSimpleExplanationInternet}.
Dadurch wird eine Kommunikation zwischen Mensch und Objekt oder Objekt und Objekt ermöglicht \cite{uckelmannArchitectingInternetThings2011}.
Solche Objekte werden auch \textit{Smart Devices} genannt.
Gartner \cite{hungGartnerInsightsHow} und statista \cite{GlobalIoTNonIoT} erwarten bis 2025 rund 30 - 75 Milliarden Geräte die mit dem Internet verbunden sind.
\\
Durch die Vernetzung alltäglicher Objekte wird unsere Umgebung klüger und reaktiv. Zum Beispiel können Kaffeemaschinen automatisch Kaffee nachbestellen oder Maschinen einen Techniker anfordern, sobald ein Defekt vorliegt.
\cite{rangerWhatIoTEverything}
\\
Die Idee, verschiedene Objekte mit dem Internet zu verbinden, wurde bereits um 1980 und 1990 behandelt, jedoch waren zu dieser Zeit die Mikrocontroller und Mikrochips zu gro{\ss} und zu langsam um effektiv eingesetzt werden zu können.
Mit der stetigen Optimierung der Chips können immer mehr Objekte mit solchen ausgestattet und somit in die Lage versetzt werden, mit ihrer Umwelt zu kommunizieren.
\cite{rangerWhatIoTEverything}
\\
Die stetige Zunahme von Smart Devices stellt jedoch aktuelle Kommunikationsprotokolle und Infrastrukturen vor neue Herausforderungen.
Diese wurden ursprünglich für statische Geräte mit einer stetigen Strom- und Internetanbindung entwickelt. Millionen Geräte, die unter anderem auch durch ein instabiles mobiles Netzwerk mit dem Internet verbunden sind, erfordern neue Protokolle und Technologien, die ihnen eine optimale Integration in die bestehende Infrastruktur ermöglichen.
\cite{uckelmannArchitectingInternetThings2011}
\newpage

\begin{comment}
- Describe the (business) domain of your work. Ask yourself: What does the common reader need to know about the domain in order to understand the results of your work? For example, if you work contributes to the claim handling process of some insurance company, describe the common claim handling process.
- Please refrain from meandering explanations about details that have no relevance for later chapters (just in order to increase your page count).
- The header "Domain / Anwendungsfeld" is supposed to be replaced or extended with some more specific header like: "Domain 'Claim handling'"
\end{comment}
