\section{Fazit}
Das Ziel der vorliegenden Arbeit war die Analyse des \ac{mqtt} Protokolls mit Hinblick auf das \textit{kluge} Verteilen der Clients an HiveMQ Nodes und die Programmierung einer Envoy Control-Plane, die einige der analysierten Features implementiert.
\\
Die Vorbereitung auf diese Analyse bedurfte einer umfangreichen Einarbeitung in das \ac{mqtt} Protokoll, um die Eigenheiten und verschiedenen Verhaltensmuster der Clients zu untersuchen.
Trotz der in den letzten Jahren stetig steigenden Verwendung des Protokolls, unter anderem bei gro{\ss}en Cloud Anbietern wie Amazon Web Services und Microsoft Azure, gibt es nur wenig veröffentlichte Informationen zu dem Betrieb von \ac{mqtt} Load-Balancern.
Auch die Unterstützung des \ac{mqtt} Protokolls in Proxies wie Nginx, HAProxy oder Envoy steht noch nicht zur Verfügung.
Bei einem Envoy Proxy ist es bereits möglich, Protokoll \textit{sniffing} für Redis, MongoDB oder Postgres zu betreiben, um eine ausführliche Analyse des Traffics durchzuführen.
Der Mangel an Ressourcen des Ökosystems \textit{load balancing \ac{mqtt}} könnte durch die Kommerzialisierung im Bereich \ac{iiot} zu begründen sein.
\\
Um \ac{iot}-Plattformen zu optimieren, müssen typischerweise entkoppelte \acl{lb} ein integraler Bestandteil des \ac{iot}-Ökosystems werden.
Sie müssen das Protokoll verstehen, Informationen der Plattform aggregieren und reaktive load balancing Entscheidungen treffen.
Nur so kann der Traffic von Millionen \ac{iot}-Geräten optimal verteilt werden.
Der \acl{lb} muss zu einem \textit{Smart-Device} im \textit{Internet of Things} werden.
\newpage

\section{Ausblick}
Die entwickelte Control-Plane macht den \acl{lb} nicht nur konfigurierbar, sondern auch programmierbar.
Dies ermöglicht eine optimale Integration des Load-Balancers durch progressive Weiterentwicklung.
Während der Bearbeitung der vorliegenden Arbeit wurde das Testwerkzeug HiveMQ Swarm \cite{teamHiveMQSwarmFind} von HiveMQ veröffentlicht. Mit diesem Tool können Client-Verhaltensmuster definiert werden, um eine HiveMQ Cluster Installation zu testen.
Mit HiveMQ Swarm kann eine Control-Plane iterativ erweitert und für den eigenen Use-Case optimiert sowie getestet werden.
\\
Mit dem Erweitern von Envoy durch \acl{wasm}-Module kann zukünftig das gesamte \ac{mqtt} Protokoll geparst werden.
Dies eröffnet neben einer Realisierung der beschriebenen \textit{Session-Affinity} die Möglichkeit, noch weitere protokollspezifische load balancing Entscheidungen zu treffen.
Mit der fünften Version des \ac{mqtt} Protokolls wurden \textit{user properties} eingeführt, die es Benutzer:innen erlauben, beliebige Daten als \textit{key-value} Paare an \ac{mqtt} Pakete anzuhängen \cite{raschbichlerMQTTHowNewa}.
Ein protokollbewusster \acl{lb} kann mit dem Protokoll \textit{sniffing} zu einem detaillierten Auswerten des \ac{mqtt} Traffics beitragen und über \textit{user properties} ein- sowie ausgehende Nachrichten mit nützlichen Informationen anreichern.
\\
Denkbar sind Erweiterungen, wie ein neues Protokoll für die Kommunikation zwischen \acl{lb} und HiveMQ Cluster.
Der \acl{lb} bietet eine \ac{mqtt} konforme Schnittstelle für die Clients und leitet die einzelnen Pakete optimiert an die Nodes des HiveMQ Clusters weiter.
Dabei können beispielsweise \verb|PUBLISH| Pakete, unter Berücksichtigung der Topic Aliasse der Clients, verbindungsunabhängig an die Nodes verteilt werden.
Die Konzepte der vorliegenden Arbeit bieten einen Einstieg, um den Nutzen derartig fundamentaler Erweiterungen für das load balancing von \ac{mqtt} Traffic zu erforschen.
