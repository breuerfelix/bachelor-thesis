\section{Fazit und Ausblick}
Das Ziel der vorliegenden Arbeit war die Analyse des \ac{mqtt} Protokolls mit Hinblick auf das \textit{kluge} Verteilen der Clients an HiveMQ Nodes und die Erstellung einer Prototyp Envoy Control Plane, die einige der analysierten Features implementiert.
\\
Die Vorbereitung auf diese Analyse bedurfte eine umfangreiche Einarbeitung in das \ac{mqtt} Protokoll um die Eigenheiten und verschiedenen Verhaltensmuster der Clients zu untersuchen. Aus diesen Untersuchungen sind beispielsweise die Testszenarien aus Kapitel \ref{ss:test} entstanden, die zeigen sollen, dass bestimmte Verhaltensmuster der Clients eine ungleiche Lastverteilung im Cluster erzeugen können. Bei der Implementierung dieser Szenarien ist aufgefallen, wie komplex und unterschiedlich die Verhaltensmuster sein können.
Unterschiedliche Client Verhaltensmuster führen jedoch nicht grundsätzlich direkt zu einem Problem bei der Lastverteilung. Es ist die Kombination von unterschiedlichen Clients mit der Eigenschaft, dass \ac{mqtt} ein zustandsbehaftetes verbindungsorientiertes Protokoll ist.
Der \acl{lb} kann nicht, wie zum Beispiel bei \ac{http}, einzelne Anfragen eines Clients an verschiedene Nodes schicken. Sobald ein Client verbunden ist, muss der Zielnode alle Anfragen des Clients bearbeiten, bis dieser die Verbindung beendet.
In der Realität will man dieses Verhalten bei \ac{http} oftmals gezielt erzwingen, um beispielsweise schnellere Antwortzeiten durch Cache-Speicher zu erzielen. Bei \ac{http} kann der \acl{lb} jedoch Pakete transparent zum Client an einen anderen Node schicken, wenn zum Beispiel der bisherige Zielnode überlastet ist. Bei \ac{mqtt} hat der \acl{lb} gar keine Möglichkeit die Last eines Clients nach dem Verbindungsaufbau zu steuern.
Dies untermauert die Relevanz der load balancing Entscheidung bei \ac{mqtt}.
\\
Aktuelle \acl{lb} wie Nginx, HAProxy und Envoy haben aufgrund der Popularität von \ac{http} viele Optimierungen 

\begin{comment}
- eigene meinung

- client transparentes roaming
- weighted cpu round robin ? worth it ?
- es gibt sehr viele verschiedene client szenarien -> hivemq swarm -> programmable lb

implementierung:
- sticky session
- mutex in der control plane
\end{comment}
