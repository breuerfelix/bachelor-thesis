\section{Architektur und Implementation}

\subsection{Envoy}
\subsubsection{Golang Data-Plane}
\subsubsection{DNS Service Discovery}
\subsubsection{Weighted Round Robin}
\subsubsubsection{HiveMQ Metriken}
\subsubsection{ClientID Network Filter}

\subsection{Deployment}
\subsubsection{HiveMQ}
\subsubsection{Data-Plane}
\subsubsection{Envoy}

\begin{comment}
- Describe the realization of your concepts, in case you have actually developed some-thing.
- Elaborate on the software architecture of your tool, in case you have developed one. Use nested UML packages, components, and interfaces in a component diagram.
- If applicable, show the deploymentof your tool in a production environment. Use UML's deployment diagram notation.
- It must be clear from the architecture, what your thesis contributes and what it takes for granted like existing systems, code bases, libraries and frameworks. For example, you can decorate the components in a UML component diagram that you have implemented and those that you just used.
- Do not delve into ordinary details by, e.g., intensively describing a "p lain old Java-object (POJO)" in all its dreary getter-setter-details. Instead pick some interesting details and de-scribe them, e.g.,
  - if   you made extensive use of a certain design pattern, describe a single concrete appli-cation of it using, e.g., UML class diagrams or
  - if your work involves a complicated conversation pattern or protocol, explained it using a UML sequence diagram, state chart, activity diagram and the like or
  - if you have developed a central and canny algorithm, you may even show its implemen-tation in, e.g., Java code.
- Show how the result of your work actually looks like. In case of a tool, provide some screenshots together with some explanatory text.
- Describe the quantity of your work, e.g., in terms of lines of code or classes etc. Please just count your own hand-crafted code but not previously existing, imported or generated code.
- Describe the quality of your work, e.g., if you have developed a large web application, run some performance tests, depicts results and draw conclusions from them.
\end{comment}
