\section{Lösungskonzept}
%nur das konzept WIE ich das lösen will zb. man kann wie clientid parsen mit einem WASM modul
%in der nächsten section zeige ich dann den eigentlichen code welcher dies tut

\subsection{Envoy}
\subsubsection{Data-Plane API}
\subsubsection{Hot Restart}

\subsection{DNS Service Discovery}

\subsection{Weighted Round Robin}
\subsubsection{Overload Protection}
\subsubsection{Client Credits}
\subsubsection{Global Tasks}

\subsection{Sticky Session with Client ID}
\subsubsection{MQTT CONNECT}
\subsubsection{Envoy WASM Network Filter}

\begin{comment}
- This is supposed to be the core of your thesis or project. Describe your work from a con-ceptual viewpoint.
- Example: In case you have developed some prototypical tool in your bachelor thesis, demonstrate how it is employed in its business context. More concrete example: Assume that your contribution is a Maven-Build-Plugin that further automates the deployment of changes to the claim handling process into production. In this case show how the plugin is integrated in the overall (continuous) integration and deployment process, which human ac-tors are involved, which external systems and so on. Elaborate on subtle edge cases you had to deal with, e.g., possible outages of external systems.
- Usedi  agramswhere appropriate. Standard notations are better than informal box-and-line-diagrams. Typical standard notations for a solution concept are
  - Business Process Modelling Notation (BPMN) or UML activity diagrams, that depict a workflow in which your tool is used
  - UML component diagrams, where your tool is represented by just a single component (without its ingredients) together with connected external systems
\end{comment}
