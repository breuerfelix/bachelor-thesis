\section{Einleitung}
\subsection{Motivation}
\subsection{Unternehmen}
\subsection{Zielsetzung}
%vielleicht im kapitel motivation?
\subsection{Aufbau der Arbeit}

\begin{comment}
- Outline your work such that readers who have not read subsequent chapters get an idea of what
  - your "problem domain" is (e.g. the software quality department of some company or the overarching research project you are working in)
  - the situation is (e.g. software tests at some company are done in complete manual fash-ion)
  - the complication / problem is (e.g. hotfixes have to be deployed totally untested in pro-duction)
  - what your approach is (e.g. establishing a concept and tool evaluation for continuously improving code coverage of automated tests)
  - what beyond the scope of your thesis is (e.g. also implementing a CICD pipeline)
  - what the actual core results are (e.g. how many tools have been evaluated)

From that it must clear, what your thesis'contribution actually is. Please delineate your contribution from every concept, method, tool, framework or whatever implementation your thesis took for granted and is built upon.

This chapter deliberately anticipates contents of the latter chapters but please refrain from spe-cific technical termsof your application domain or solution here. Trade technical accuracy for com-mon comprehensibility here. Write this chapter as a "management summary" and do not spend more than two to three pages here.

Also explain structure of your thesis, i.e., summarize the contents of each chapter in one sentence or paragraph and describe the overarching storyline, i.e., how chapters build upon each other.
\end{comment}
